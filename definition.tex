\documentclass[a4paper,14pt]{article}

\usepackage{cmap}                       % поиск в PDF
\usepackage{mathtext}                   % русские буквы в фоhмулах
\usepackage[T2A]{fontenc}               % кодировка
\usepackage[utf8]{inputenc}             % кодировка исходного текста
\usepackage[english,russian]{babel}     % локализация и переносы

\usepackage{amsmath,amsfonts,amssymb,amsthm,mathtools}
\usepackage{icomma}

\newcommand*{\hm}[1]{#1\nobreak\discretionary{}
{\hbox{$\mathsurround=0pt #1$}}{}}

\mathtoolsset{showonlyrefs=true}

\usepackage{geometry}
\geometry{top=15mm}
\geometry{bottom=15mm}
\geometry{left=15mm}
\geometry{right=15mm}

\everymath{\displaystyle}

% Колонтитулы
\usepackage{fancyhdr}
\pagestyle{fancy}
\fancyhf{}
\rhead{Определение}
\lhead{2020}
\begin{document}
\section*{Определение}
Представим бесконечную последовательность чисел $\left\{a_i\right\}_{i=0}^\infty$. Можно записать ее
в качестве коэффициентов бесконечного полинома $f(t)$:
\[
	f(t) = \sum_{k=0}^{\infty} a_kt^k = a_0 + a_1t + a_2t^2 + \ldots
\]
$f(t)$ является производящей функцией. \\
\textbf{NB}: $f(t)$ является формальным степенным рядом, поэтому не обязательно, что представленный
ряд будет сходящимся. Нас интересуют коэффициенты, нежели чем переменные. \\
Самый простой пример производящей функции – геометрическая прогрессия:
\begin{align*}
	&a_0 = 1, a_n = qa_{n-1} \\
	1 + qt + &q^2t^2 + q^3t^3 + \ldots = \frac{1}{1 - qt}
.\end{align*}
Это верно, потому что:
\[
	(1 - qt)(1 + qt + q^2t^2 + \ldots) = 1 + qt - qt + q^2t^2 - q^2t^2 + \ldots = 1.
\] 
\end{document}
